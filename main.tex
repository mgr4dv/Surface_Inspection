\documentclass{article}
\usepackage[utf8]{inputenc}
\usepackage{verbatim}

\title{Autonomous Surface Inspection and Adaption on a Mobile Robotics Platform}
\author{Mark Restrepo, Edward Bruce, Ryan Rosenfeld, Will Gerard, Atiena Branch, Gregory Lewin, Nicola Bezzo}
\date{January 2018}

\begin{document}

\section{Abstract}

% Submission of 200-500 word abstracts for contributed papers not previously published or presented. Abstracts should explicitly address the following:

% The design problem being solved
% Background/relevant literature
% Methodological approach developed or adopted
% Preliminary results and implications

Autonomous mobile robots  currently serve a wide array of uses -- from mapping remote locations to planetary exploration. However, different terrain and surface properties can affect their performance. As a result, accurate identification of surfaces and their properties is vital for these robots to achieve optimal performance. Most current solutions to this problem take a supervised approach. While these supervised learning approaches have been successful, they have several disadvantages. Robots must be trained offline before running, and are unable to adapt to terrains that were not present in the training data. Our approach is different: using unsupervised learning, our algorithm allows a robot to identify surfaces it has never seen before and adjust its operation for optimal performance -- without any training whatsoever. Data from a LIDAR sensor, 3D Camera, and IMU are combined to create vectors passed to a modified density-based spatial clustering of applications with noise (DBSCAN) algorithm. This algorithm performs online, unsupervised surface classification, learning new surfaces as it travels and storing each surface's ideal operating parameters. The algorithm also discards outliers to prevent overfitting the data. When the robot detects that it has moved onto a new surface it increases velocity while maintaining IMU vibrations below a specific threshold, gradually learning the optimal traversal parameters for the surface. The algorithm has been tested on several surfaces with an autonomous ground vehicle. These tests have validated that the algorithm can successfully distinguish between most surfaces and drive optimally through them while carrying a fragile load reliant on limited vibration, without the use of any a priori training data. This unsupervised learning approach will allow for more robust autonomous operation on unknown surfaces. 

\end{document}

Greg?s comments:

\begin{itemize}
\item Add Bezzo and me as authors. Nicola should be last, since he?s the inspiration behind the work.
\item You focus early on identification of surfaces, but ultimately, your goal is to operate optimally on any given surface. As presented today, this is done as a combination of surface recognition and performance testing. This should be reflected early in the abstract.
\item ?Our approach...?: you should note that you are using an unsupervised approach
\item After ?...new surface...? note that values are stored so that the terrain doesn?t have to be learned again (with an asterisk in the paper that reflects today?s discussion)
\item You need a wrap-up sentence at the end...something to reiterate that this will allow autonomous robots to perform better in unknown environments, for example.
\end{itemize}